%Gran parte de esto lo generó el LyX
\documentclass[12pt,spanish]{article}
\usepackage{lmodern}
\usepackage[T1]{fontenc}
\usepackage[utf8]{inputenc}
\usepackage[a4paper]{geometry}
\geometry{verbose,tmargin=2cm,bmargin=2cm,lmargin=70pt,rmargin=70pt,headheight=35pt}
\usepackage{fancyhdr}
\pagestyle{fancy}
\usepackage{float}
\usepackage{textcomp}
\usepackage{graphicx}

%Manejo de notas al pie; Captions en negrita en Figuras y Tablas
\usepackage[bottom]{footmisc}
\usepackage[hang,bf]{caption}
\usepackage{babel}
\addto\shorthandsspanish{\spanishdeactivate{~<>}}

\begin{document}

%Encabezado
\lhead{\includegraphics[width=3cm]{logo-facu.jpg}}
\chead{}
\rhead{\textsc{Análisis numérico I} - 2do cuatrimestre de 2013\\
TP Nº1 - Ávila Alterach (94950), Fernandez Lema (93410)}

%Carátula
\begin{titlepage}
\thispagestyle{empty} %No poner número de página en título

\begin{center}
	\includegraphics[width=10cm]{logo-facu-grande.jpg}
	\vfill
	
	\huge{75.12 - Análisis Numérico I} \\
	\LARGE{Trabajo Práctico Nº1 \\ Resolución de Sistemas de Ecuaciones Lineales}
	\vfill
	
	\normalsize{
	\begin{tabular}{lll}
		\textbf{Nombre} & \textbf{Correo electrónico} & \textbf{Padrón} \\ \hline 
		Gonzalo Ávila Alterach & gonzaloavilaalterach@gmail.com & 94950 \\
		Nicolás Mariano Fernandez Lema & nicolasfernandezlema@gmail.com & 93410 \\
	\end{tabular}
	}
	\vfill
		
	\large{Fecha de entrega: 16 de octubre \\ 2º cuatrimestre de 2013}
	\vfill
\end{center}
\end{titlepage}
\pagebreak

\section{Resumen}
El sistema a resolver se trata de $Ax=b$ , siendo A una matriz cuadrada tridiagonal, de dimensiones $(n-1)*(n-1)$.
El vector incógnita x a averiguar está formado por los elementos $(c_1, c_2, ..., c_{n-1})$, ya que sabiendo los $c_k$ se pueden hallar el resto de los coeficientes de los distintos polinomios.
Debido a las ecuación 4 y la 7, se puede apreciar que la matriz A es tridiagonal, y sus elementos son lo siguientes:
Además, debido a que los datos de entrada utilizados la diferencia entre los $\theta _k$consecutivos es constante, la matriz A queda simétrica.

\end{document}
