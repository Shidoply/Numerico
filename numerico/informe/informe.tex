%Gran parte de esto lo generó el LyX
\documentclass[12pt,spanish]{article}
\usepackage{lmodern}
\usepackage[T1]{fontenc}
\usepackage[utf8]{inputenc}
\usepackage[a4paper]{geometry}
\geometry{verbose,tmargin=2cm,bmargin=2cm,lmargin=70pt,rmargin=70pt,headheight=35pt}
\usepackage{fancyhdr}
\pagestyle{fancy}
\usepackage{float}
\usepackage{textcomp}
\usepackage{graphicx}
\usepackage[spanish]{babel}
\usepackage{amsmath}

%Manejo de notas al pie; Captions en negrita en Figuras y Tablas
\usepackage[bottom]{footmisc}
\usepackage[hang,bf]{caption}
\usepackage{babel}
\addto\shorthandsspanish{\spanishdeactivate{~<>}}

%Incluir librería para listado código
\usepackage{xcolor,listings}

\lstset{
  literate={á}{{a}}1
           {é}{{e}}1
           {í}{{i}}1
           {ó}{{i}}1
           {ú}{{i}}1
}

%fuentes y colores
\lstset{
 belowcaptionskip=1\baselineskip,
  breaklines=true,
  frame=L,
  language=C,
  showstringspaces=false,
  basicstyle=\ttfamily\footnotesize,
  keywordstyle=\bfseries\color{green!30!black},
  commentstyle=\itshape\color{purple!30!black},
  identifierstyle=\color{blue},
  stringstyle=\color{orange},
  flexiblecolumns=true,
  numbers=left,                    % where to put the line-numbers; possible values are (none, left, right)
  numbersep=10pt,                   % how far the line-numbers are from the code
  numberstyle=\footnotesize, % the style that is used for the line-numbers
  tabsize=4,
}

\begin{document}

%Encabezado
\lhead{\includegraphics[width=3cm]{logo-facu.jpg}}
\chead{}
\rhead{\textsc{Análisis numérico I} - 2do cuatrimestre de 2013\\
TP Nº1 - Ávila Alterach (94950), Fernandez Lema (93410)}

%Carátula
\begin{titlepage}
\thispagestyle{empty} %No poner número de página en título

\begin{center}
	\includegraphics[width=10cm]{logo-facu-grande.jpg}
	\vfill
	
	\huge{75.12 - Análisis Numérico I} \\
	\LARGE{Trabajo Práctico Nº1 \\ Resolución de Sistemas de Ecuaciones Lineales}
	\vfill
	
	\normalsize{
	\begin{tabular}{lll}
		\textbf{Nombre} & \textbf{Correo electrónico} & \textbf{Padrón} \\ \hline 
		Gonzalo Ávila Alterach & gonzaloavilaalterach@gmail.com & 94950 \\
		Nicolás Mariano Fernandez Lema & nicolasfernandezlema@gmail.com & 93410 \\
	\end{tabular}
	}
	\vfill
		
	\large{Fecha de entrega: 16 de octubre \\ 2º cuatrimestre de 2013}
	\vfill
\end{center}
\end{titlepage}
\pagebreak

\section*{Resumen}
En este problema, el sistema a resolver se trata de $Ax=b$ , siendo A una matriz cuadrada tridiagonal, de dimensiones $(n-1)^2$.

El vector incógnita $x$ a averiguar está formado por los elementos $(c_1, c_2, ..., c_{n-1})$, ya que sabiendo su valor se pueden hallar el resto de los coeficientes de todos los polinomios.
Debido a las ecuación 4 y la 7, se puede apreciar que la matriz A es tridiagonal, y sus elementos son lo siguientes:

\[
A_n=
  \begin{pmatrix}
   2(h_0+h_1) & h_1 & 0 &  0 & \cdots & 0 & 0 &\\
   h_1 & 2(h_1+h_2) & h_2 & 0 &  \cdots & 0 & 0 &\\
   0 & h_2 & 2(h_2+h_3) & h_3  &  \cdots & 0 & 0 &\\
   \vdots  & \vdots & \vdots & \vdots & \ddots & \vdots & \vdots &\\
   0 & 0 & 0 & 0 & \cdots & h_{n-1} & 2(h_{n-1}+h_n)\\
  \end{pmatrix}
\]

Además, debido a que los datos de entrada utilizados la diferencia entre los $\theta _k$ consecutivos es constante, los $h_k$ también lo son, y por lo tanto la matriz queda simétrica.

Los programas se desarrollaron para resolver matrices genéricas, sin tener en cuenta que los sistemas del problema a resolver son tridiagonales, por lo que es una resolución ineficiente: se esta utilizando más memoria que la necesaria y también se están multiplicando muchas veces ceros por ceros.

\section*{Polinomios encontrados}
Las soluciones encontradas mediante Jacobi fueron las siguientes:

\section*{Radio espectral de la matriz de Jacobi ($\rho_J$)}
\section*{Radio espectral de la matriz de Gauss-Seidel ($\rho_{GS}$)}
\section*{Gráfico de $\omega$ en función de iteraciones hasta convergencia}
\section*{Gráfico de splines obtenidos}
\section*{Máxima velocidad de avance}
\section*{Viviendas alcanzadas por el fuego}
\section*{Conclusiones}

\newpage
\section{Código (C++)}
\lstinputlisting[language=C++,texcl=true]{../tp1.cpp}

\end{document}
